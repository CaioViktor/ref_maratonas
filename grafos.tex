\section{Grafos}

\subsection{Global Min Cut}
Menor cut para desconectar um grafo conectado. Vers\~{a}o $O(V^3)$, usando lista de adjac\^{e}ncias
\Cpp{\detokenize{grafos/flow/stoer_wagner_n3}}
% Problema: http://community.topcoder.com/stat?c=problem_statement&pm=7246&rd=10658
\divisor

\subsection{Unweighted General Matching}
Algoritmo de Edmond para Max Matching em um grafo n\~{a}o orientado e sem pesos.
Complexidade: $O(V^4)$.

\Cpp{\detokenize{grafos/edmond_blossom}}

\divisor

\subsection{Grafos Bipartidos}

\subsubsection{MCMB (Hopcroft Karp)}
Complexidade: $O(\sqrt{V}*E)$
\Cpp{\detokenize{grafos/hopcroft_karp}}
\divisor

\subsubsection{Maximum Independent Set (MIS)}
Um conjunto independente de um grafo G é um conjunto de vértices onde nenhum deles tem uma aresta ligando a outro.

Num grafo bipartido: $MIS = V - MCMB$
\divisor

\subsubsection{Min Vertex Cover (Teorema de König)}
Num grafo bipartido: $Min Vertex Cover = MCMB$

Observação: Em um grafo bipartido, o conjunto \textbf{menor é o Min Vertex Cover} e o conjunto \textbf{maior é o Max Independent Set}.
\divisor

\subsubsection{Max Weighted Independent Set}
Em um grafo bipartido, ao se adicionar pesos aos v\'{e}rtices, temos que o Max Weighted Independent Set \'{e} o set com a maior soma dos pesos dos v\'{e}rtices, tal que n\~{a}o haja uma aresta entre dois v\'{e}rtices deste set.

Pode-se resolver com Max flow: Adiciona um Source e um Sink. Nas arestas do grafo original, coloca-se $cap=\infty$. Nas arestas vindo da source, coloca-se o peso dos v\'{e}rtices correspondentes e faz-se o mesmo para as arestas chegando em sink.

O Max Weighted Independent Set ser\'{a} a soma dos pesos dos v\'{e}rtices menos o Max Flow obtido do grafo.
% problema: http://www.codechef.com/problems/TWOCOMP
\divisor

\subsection{MCMF}
Edmonds Karp com D\"{i}jkstra. Complexidade: $O(V*E*max(Maior cost,Maior cap))$
% problema: https://icpcarchive.ecs.baylor.edu/index.php?option=com_onlinejudge&Itemid=8&category=569&page=show_problem&problem=4277
\Cpp{\detokenize{grafos/flow/mcmf}}
\divisor

\subsection{Dinic}
Complexidade: $O(V^2*E)$
\Cpp{\detokenize{grafos/flow/dinic}}
\divisor

\subsection{Interse\c{c}\~{a}o de caminhos em \'{a}rvores}
Tendo calculado o LCA, usa-se o m\'{e}todo para verificar se dois caminhos t\^{e}m pelo menos um v\'{e}rtice em comum.
\Cpp{\detokenize{grafos/tree_path_intersect}}
\divisor
