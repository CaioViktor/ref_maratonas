\section{Geometria}

% BASICO % % % % % % % % % % % % % % %
\subsection{Template b\'{a}sico}
\Cpp{\detokenize{geometria/basico}}

% TEOREMAS % % % % % % % % % % % % % %
\subsection{Teoremas Importantes}

\subsubsection{Teorema de Pick}
Teorema para calcular a \'{a}rea de um pol\'{i}gono simples em que os v\'{e}rtices são n\'{u}meros inteiros no plano XY. A \'{a}rea de tal pol\'{i}gono é: 
\\\centerline{A(b) = I + $\frac{B}{2}$ - 1}
\\Onde I \'{e} o n\'{u}mero de pontos inteiros interiores ao pol\'{i}gono e B o n\'{u}mero de pontos inteiros na borda.
\divisor


% POLIGONOS % % % % % % % % % % % % % %

\subsection{Pol\'{i}gonos}
\subsubsection{Pol\'{i}gonos regulares}
S\~{a}o pol\'{i}gonos equil\'{a}teros e equiangulares. Podem ser convexos ou estrelas. No caso dos convexos, quando $n\rightarrow\infty$, tal pol\'{i}gono \'{e} um c\'{i}rculo.
\\ A partir de 3 v\'{e}rtices, \'{e} poss\'{i}vel encontrar todos os outros do pol\'{i}gono regular, j\'{a} que todos estar\~{a}o na circunfer\^{e}ncia definida por tais pontos. 
\divisor

\subsubsection{Convex Hull por Monotone Chain}
Os vetores "up" e "down" s\~{a}o, respectivamente, o convex hull superior e inferior.
\\ \textbf{Importante}: Caso n\~{a}o use o set, lembrar de dar sort nos pontos.
\Cpp{\detokenize{geometria/poligonos/monotone_chain}}
\divisor

\subsubsection{\'{A}rea de pol\'{i}gono simples}
\Cpp{\detokenize{geometria/poligonos/area_polygon}}
\divisor

\subsubsection{Centroide'}
\Cpp{\detokenize{geometria/poligonos/centroide}}
Para uma distribuição uniforme de massa, o centro d massa do pol\'{i}gono coincide com o centroide.
\divisor

% ROTATING CALIPERS % % % % % % % % % % % % % % %

\subsection{Rotating Calipers}

T\'{e}cnica para achar os pares de pontos antipodais. Primeiro calcula-se a convex hull do set para ent\~{a}o aplicar o Rotating Calipers.
\\ Lembrar: Para encontrar um bom comparativo dos \^{a}ngulos, ignorar a rota\c{c}\~{a}o do caliper, fixar a refer\^{e}ncia e comparar um \^{a}ngulo theta entre os vetores. Quando a rela\c{c}\~{a}o envolver 90 graus, usar produto escalar, quando envolver 180 graus, usar vetorial (ver exemplo de bounding box).

\subsubsection{Di\^{a}metro de um set de pontos}
Exemplo mais b\'{a}sico de rotating calipers. up e dn v\^{e}m do Monotone Chain.
\Cpp{\detokenize{geometria/rotating_calipers/polygon_diam}}
% problema: SPOJ TFOSS, UVa 1111 (nao imprimir ultimo \n no do UVa)
\divisor

\subsubsection{Minimum enclosing rectangle}
Basta usar 4 calipers (diferente das 2 do problema do di\^{a}metro). Tomar cuidado com a compara\c{c}\~{a}o dos \^{a}ngulos.
\\ O menor ret\^{a}ngulo (tanto em \'{a}rea quanto em per\'{i}metro) sempre tem uma das arestas em comum com uma aresta do pol\'{i}gono.
\Cpp{\detokenize{geometria/rotating_calipers/min_enclosing_rect}}
% problema: UVa 12307
\divisor

\subsubsection{Maior dist\^{a}ncia entre pol\'{i}gonos convexos}
Considerando que a maior dist\^{a}ncia \'{e} o par de v\'{e}rtices mais dist\^{a}ntes entre si, o problema \'{e}  resolvido usando um caliper em cada pol\'{i}gono e tratando tal dist\^{a}ncia como um di\^{a}metro de um pol\'{i}gono convexo.
\divisor

\subsubsection{Bridges entre pol\'{i}gonos convexos (convex hull merge)}
Colocando um caliper em cada pol\'{i}gono, existir\~{a}o duas bridges, que s\~{a}o pares de pontos co-podais (pontos em que se pode tra\c{c}ar uma reta em cada, de forma que elas sejam paralelas e todos os pontos de ambos os sets estejam em somente um dos lados).
\\ Mais f\'{a}cil gerar o convex hull novamente, em $O(n \log{n})$, apesar de a t\'{e}cnica dos rotating calipers ser mais r\'{a}pida ($O(n)$);
\divisor

% FUNÇÕES BÁSICAS % % % % % % % % % % % % % % %

\subsection{Fun\c{c}\~{o}es b\'{a}sicas}
\subsubsection{Quadrante}
Fun\c{c}\~{a}o que retorna o quadrante em que o ponto est\'{a}. $o$ ser\'{a} a origem do plano cartesiano de refer\^{e}ncia.
\Cpp{\detokenize{geometria/basico/quadrante}}
\divisor

\subsubsection{Proje\c{c}\~{a}o de vetor}
Calcula m\'{o}dulo da proje\c{c}\~{a}o de um vetor b em cima do vetor a.
\Cpp{\detokenize{geometria/basico/proj}}
\divisor

\subsubsection{\^{A}ngulo agudo}
\Cpp{\detokenize{geometria/basico/acute}}

% SEGMENTOS, LINHAS PONTOS % % % % % % % % % % %

\subsection{Linha, segmentos de reta e pontos}

\subsubsection{Pontos em lattice}
\Cpp{\detokenize{geometria/segmentos_e_linhas/pts_in_lattice}}
\divisor

% SWEEPLINE % % % % % % % % % % % % % % % % % % 

\subsection{Sorting}
\subsubsection{Sort por \^{a}ngulo}
Comparador para realizar sort por \^{a}ngulo. Bom porque n\~{a}o usa opera\c{c}\~{o}es com ponto flutuante.
\Cpp{\detokenize{geometria/aleatorio/angle_sort}}
%problema: UVa 12675 (Brasileira 2013)
\divisor

\subsubsection{Sort polar de segmentos de reta}
Sort funciona quando \'{e} poss\'{i}vel tra\c{c}ar uma semi-reta partindo de "origin" e cortando ambos os segmentos.
\Cpp{\detokenize{geometria/segmentos_e_linhas/point_segment_sort}}
\divisor

% TRIGONOMETRIA % % % % % % % % % % % % % % % %

\subsection{Trigonometria}

\subsubsection{Lei dos cossenos}

$a^2 = b^2 + c^2 - 2*b*c*cos(\hat{A})$
\\$b^2 = a^2 + c^2 - 2*a*c*cos(\hat{B})$
\\$c^2 = a^2 + b^2 - 2*a*b*cos(\hat{C})$

Onde $\hat{X}$ \'{e} o \^{a}ngulo oposto ao lado de tamanho $x$.
\divisor

\subsubsection{Lei dos senos}
$\frac{a}{sin(\hat{A})} = \frac{b}{sin(\hat{B})} = \frac{c}{sin(\hat{C})}$
\divisor
