\section{Geometria}

\subsection{Teoremas Importantes}

\subsubsection{Teorema de Pick}
Teorema para calcular a \'{a}rea de um pol\'{i}gono simples em que os v\'{e}rtices são n\'{u}meros inteiros no plano XY. A \'{a}rea de tal pol\'{i}gono é: 
\\\centerline{A(b) = I + $\frac{B}{2}$ - 1}
\\Onde I \'{e} o n\'{u}mero de pontos inteiros interiores ao pol\'{i}gono e B o n\'{u}mero de pontos inteiros na borda.

\divisor

\subsection{Convex Hull por Monotone Chain}
Os vetores "up" e "down" s\~{a}o, respectivamente, o convex hull superior e inferior.
\\ \textbf{Importante}: Caso n\~{a}o use o set, lembrar de dar sort nos pontos.
\Cpp{\detokenize{geometria/monotone_chain}}

\divisor

\subsection{Rotating Calipers}

T\'{e}cnica para achar os pares de pontos antipodais. Primeiro calcula-se a convex hull do set para ent\~{a}o aplicar o Rotating Calipers.
% Em construção: Faltam outros problemas
\subsubsection{Di\^{a}metro de set de pontos}
Exemplo mais b\'{a}sico de rotating calipers. up e dn v\^{e}m do Monotone Chain.
\Cpp{\detokenize{geometria/rotating_calipers/polygon_diam}}
%problema: http://www.spoj.com/problems/TFOSS/
\divisor
