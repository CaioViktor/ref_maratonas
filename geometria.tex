\section{Geometria}

\subsection{Teoremas Importantes}

\subsubsection{Teorema de Pick}
Teorema para calcular a \'{a}rea de um pol\'{i}gono simples em que os v\'{e}rtices são n\'{u}meros inteiros no plano XY. A \'{a}rea de tal pol\'{i}gono é: 
\\\centerline{A(b) = I + $\frac{B}{2}$ - 1}
\\Onde I \'{e} o n\'{u}mero de pontos inteiros interiores ao pol\'{i}gono e B o n\'{u}mero de pontos inteiros na borda.

\divisor

\subsection{Convex Hull por Monotone Chain}
Os vetores "up" e "down" s\~{a}o, respectivamente, o convex hull superior e inferior.
\\ \textbf{Importante}: Caso n\~{a}o use o set, lembrar de dar sort nos pontos.
\Cpp{\detokenize{geometria/monotone_chain}}

\divisor

\subsection{Rotating Calipers}

T\'{e}cnica para achar os pares de pontos antipodais. Primeiro calcula-se a convex hull do set para ent\~{a}o aplicar o Rotating Calipers.

\subsubsection{Di\^{a}metro de um set de pontos}
Exemplo mais b\'{a}sico de rotating calipers. up e dn v\^{e}m do Monotone Chain.
\Cpp{\detokenize{geometria/rotating_calipers/polygon_diam}}
% problema: SPOJ TFOSS, UVa 1111 (nao imprimir ultimo \n no do UVa)
\divisor

\subsubsection{Minimum enclosing rectangle}
Basta usar 4 calipers (diferente das 2 do problema do di\^{a}metro).
% problema: UVa 12307
\divisor

\subsubsection{Maior dist\^{a}ncia entre pol\'{i}gonos convexos}
Considerando que a maior dist\^{a}ncia \'{e} o par de v\'{e}rtices mais dist\^{a}ntes entre si, o problema \'{e}  resolvido usando um caliper em cada pol\'{i}gono.
\divisor

\subsubsection{Bridges entre pol\'{i}gonos convexos (convex hull merge)}
Colocando um caliper em cada pol\'{i}gono, existir\~{a}o duas bridges, que s\~{a}o pares de pontos co-podais (pontos em que se pode tra\c{c}ar uma reta em cada, de forma que elas sejam paralelas e todos os pontos de ambos os sets estejam em somente um dos lados).
\\ Mais f\'{a}cil gerar o convex hull novamente, em $O(n \log{n})$.
\divisor

\subsection{Fun\c{c}\~{o}es b\'{a}sicas}
\subsubsection{Quadrante}
Fun\c{c}\~{a}o que retorna o quadrante em que o ponto est\'{a}. $o$ ser\'{a} a origem do plano cartesiano de refer\^{e}ncia.
\Cpp{\detokenize{geometria/basico/quadrante}}
\divisor
