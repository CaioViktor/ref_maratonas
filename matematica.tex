\section{Matemática}

% % % % % % % % % % % % %
% % TEORIA DOS NÚMEROS 
% % % % % % % % % % % % %
\subsection{Teoria dos Números e Matemática Discreta}


\subsubsection{Multiplicative/Modular Inverse}
Encontra X tal que a $\times$ x = $\frac{a}{b}$ (mod m), ou seja, x é o equivalente do inverso do módulo de b.
\Cpp{\detokenize{matematica/teoria_dos_numeros/modular_inverse}}
\divisor

\subsubsection{Exponenciação modular}
Calcula $a^b$ (mod m) em tempo O($\log{b}$)
\Cpp{\detokenize{matematica/teoria_dos_numeros/modpow}}
\divisor


% % % % % % % % % % % % %
% % CÁLCULO
% % % % % % % % % % % % %

\subsection{Cálculo}

\subsubsection{Integral definida/Regra de Simpson}
Integra $f(x)$ de $a$ a $b$. Necessário implementar $f(x)$.
\Cpp{matematica/calculo/simpson}  
% Problema bom pra testar: UVA 12528 - Environment Protection (Nacional 2012)

\divisor